\documentclass{article}
\usepackage[margin=1in]{geometry}

\title{Biology, Theme 1 Chapter 3 Activity 1}
\author{Clifton Toaster Reid}
\date{December 2023}

\begin{document}
\maketitle

\section{Modélisation du climat}

\subsection{Étapes de modélisation}

La modélisation du climat se déroule en plusieurs étapes:

\begin{enumerate}
\item \textit{Collecte des données}: Les scientifiques collectent des données issues du terrain, des données paléoclimatiques ou encore des observations satellitaires.
\item \textit{Modélisation}: Les données récoltées sont ensuite intégrées dans des équations des lois fondamentales. La planète est étudiée portion par portion selon une grille à volume défini.
\item \textit{Codage}: Les informations obtenues avec les équations sont ensuite codées.
\item \textit{Validation}: Le modèle obtenu à l'échelle globale est ensuite comparé aux données actuelles et paléoclimatiques.
\item \textit{Évaluation}: Le modèle est ensuite évalué en fonction de sa précision et de sa capacité à reproduire les observations.
\item \textit{Diffusion}: Les résultats du modèle sont ensuite diffusés à la communauté scientifique et au grand public.
\end{enumerate}

\subsection{Activités humaines et changement climatique}

Les activités humaines sont à l'origine de l'augmentation du taux de gaz à effet de serre (GES) atmosphérique, et donc du changement climatique global. Ces activités comprennent:

\begin{itemize}
\item La combustion des combustibles fossiles (charbon, pétrole, gaz naturel) pour produire de l'énergie.
\item La production de ciment.
\item La déforestation et la réhabilitation des sols.
\item Les fuites accidentelles de gaz naturel.
\item La fermentation dans l'estomac des ruminants.
\item La fermentation dans les décharges.
\end{itemize}

Ces activités humaines émettent des GES dans l'atmosphère, qui absorbent et retiennent la chaleur du Soleil. Cela entraîne un réchauffement de la planète, ce que l'on appelle le changement climatique global.

\subsection{Le GIEC et ses travaux}

Le Groupe d'experts intergouvernemental sur l'évolution du climat (GIEC) est une organisation intergouvernementale créée en 1988 par le Programme des Nations unies pour l'environnement (PNUE) et l'Organisation météorologique mondiale (OMM). Il est composé de milliers de scientifiques du monde entier, qui se réunissent tous les six ans pour rédiger un rapport sur l'état des connaissances scientifiques sur le changement climatique.

Les travaux du GIEC permettent de mieux comprendre les causes et les conséquences du changement climatique, et d'élaborer des politiques pour y faire face.

\subsection{Scénarios climatiques}

Les scientifiques du GIEC établissent différents scénarios climatiques pour l'avenir, qui dépendent des émissions de gaz à effet de serre. Les scénarios les plus pessimistes prévoient une augmentation de la température moyenne globale de 5°C d'ici la fin du siècle, tandis que les scénarios les plus optimistes prévoient une augmentation de 1,5°C.

Les changements climatiques envisagés par ces scénarios ont des conséquences importantes sur l'environnement et les sociétés humaines.

\subsection{Considérations finales}

Le changement climatique est un défi majeur pour l'humanité. Il est important de prendre des mesures pour réduire les émissions de gaz à effet de serre et pour s'adapter aux changements climatiques déjà en cours.

\end{document}
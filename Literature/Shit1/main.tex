\documentclass{article}
\usepackage[margin=2cm]{geometry}
\author{Clifton Toaster Reid}
\title{literatura laboro}

\begin{document}
\maketitle
\tableofcontents
\pagebreak

\section{Questions}

\subsection*{1- Quel est le nom de la méthode qu’utilise les scientifiques ?}

La démarche scientifique

\subsection*{2- De quoi sont convaincus les savants de l ‘Antiquité ? Donnez un des noms de ceux-ci.}

Les savants de l'Antiquité, comme Aristote ou Ptolémée, sont convaincus que la Terre est immobile au centre de l'univers et que le Soleil, comme tous les astres, tourne autour d'elle.

\subsection*{3- Quel est le nom de l’hypothèse défendue par ces savants ?}

L'hypothèse du géocentrisme

\subsection*{4- Comment se fait toutes les observations à l’époque ?}

À l'époque, toutes les observations se font à l'œil nu.

\subsection*{5- Comment qualifie-t-on une impression fondée sur l’observation ?}

On qualifie une impression fondée sur l'observation d'intuition.

\subsection*{6- Qu’observent les astronautes, qui les conduit à ajouter l’épicycle ?}

Les astronautes observent que le mouvement des planètes forme parfois une boucle dans le ciel, ce qui est incompatible avec un mouvement circulaire autour de la Terre.

\subsection*{7- Quelle est la seconde hypothèse avancée et de qui est-elle ?}

La seconde hypothèse avancée est l'hypothèse héliocentrique, selon laquelle le Soleil est au centre de l'univers et toutes les planètes, dont la Terre, tournent autour de lui. Elle est avancée par l'astronome Nicolas Copernic au 15e siècle.

\subsection*{8- Quelle est la première étape de la démarche scientifique ?}

La première étape de la démarche scientifique est la formulation d'une hypothèse.

\subsection*{9- Quelle est la seconde étape ?}

La seconde étape de la démarche scientifique est la construction d'un modèle.

\subsection*{10- Quelle est le nom de la seconde hypothèse ?}

La seconde hypothèse est l'hypothèse héliocentrique.

\subsection*{11- Quelle est la troisième étape ? Et qu’est-ce qui y est ajouté plus tard ?}

La troisième étape de la démarche scientifique est la validation de l'hypothèse. Elle consiste à tester l'hypothèse par des observations ou des expériences. Plus tard, la troisième étape sera complétée par la révision par les pairs, qui consiste à soumettre les résultats des recherches à un jury d'experts pour validation.

\subsection*{12- Qu’est-ce qui fait qu’une hypothèse est valide ?}

Une hypothèse est valide si elle résiste à l'épreuve des observations et des expériences.

\subsection*{13- En quoi consiste la démarche scientifique ?}

La démarche scientifique consiste à formuler une hypothèse, à construire un modèle, à valider l'hypothèse et à publier les résultats.

\subsection*{14- En quoi consiste > la revue par les pairs > ?}

La revue par les pairs consiste à soumettre les résultats des recherches à un jury d'experts pour validation. Cela permet de garantir la qualité des recherches et de prévenir la publication de résultats erronés ou biaisés.

\pagebreak
\section*{Acknowledgements}
\addcontentsline{toc}{section}{Acknowledgements}

For the help provided by automation and software develloped by other talented devellopers than myself, I would like to thank the great people bellow, for everything they did, helping me gain time on unessessary work.
This work include,
\begin{itemize}
  \item Improving text quality
  \item Accelerating video and audio understanding by transcription
  \item Accelerating text understanding by removing parasitive information
\end{itemize}

All of this is possible thanks to the great teams of 
\begin{itemize}
  \item OpenAI
  \begin{itemize}
    \item \Large \textbf{Whisper} 
    
    Transcription and translations of audio based documents.
  \end{itemize}
  \item Google DeepMind
  \begin{itemize}
    \item \Large \textbf{Bard — PaLM2}
    
    Manipulation of textual and image based information. Including 
    \begin{itemize}
      \item Summarize text based information
      \item Increase text quality
    \end{itemize}
  \end{itemize}
  \item LanguageTooler GmbH
  \begin{itemize}
    \item LanguageTool — Self-hosted server
    
    Increasing text quality, using the following methods.
    \begin{itemize}
      \item Correcting typing errors
      \item Correcting grammatical errors
      \item Correcting text's structure
    \end{itemize}
  \end{itemize}
\end{itemize}

For all their work I thank thel again.

\end{document}